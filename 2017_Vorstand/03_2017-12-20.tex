\begin{Protokoll}{
        Sitzungsleitung                 = {LuG},             % Kürzel wie in Hauptdatei definiert
        Protokollfuehrer                = {MiL},              % Kürzel wie in Hauptdatei definiert
        Sitzungszeit/Datum              = {01.11.2017},       % \
        Sitzungszeit/BeginnZeit         = {20:13},            %  > Die Verwendung dieser Variablen 
        Sitzungszeit/EndeZeit           = {21:56},            % /   wird im Header konfiguriert!
        Ort                             = {Mumble},            % oder jeder andere Ort
        Aufzeichnung/URL                = {},                % gerne auch leer lassen
        Aufzeichnung/Verantwortlicher   = {},                  % Kürzel wie in Hauptdatei definiert
        Status                          = {false},            % genehmigt oder nicht (boolean)
    }
    
    \begin{Anwesenheitsliste}
        \anwesenheit{JoH}{e}
        \anwesenheit{LuG}{1}
        \anwesenheit{FeW}{1}
        \anwesenheit{ElH}{e}
        \anwesenheit{MiL}{1}
    \end{Anwesenheitsliste}
    
    \textbf{Der Vorstand ist beschlussfähig.}
    
    \genehmigeLetztesProtokoll{true}
    % https://github.com/JungePiratenDeutschland/Protokoll/blob/master/2017_Vorstand/Protokoll.pdf
    
    
    \TOP{Tätigkeitsberichte der Vorstandsmitglieder}
    \begin{description}
        \item[\getPersonAsWebsiteLink{JoH}:] \emph{abwesend}
        \item[\getPersonAsWebsiteLink{LuG}:] \     
        \begin{itemize}
            \item Telefonate, E-Mail-Kontakte
            \item GO-Vorschlag erstellt
            \item Wiki- und Trello-Zugang beantragt
            \item Trello aufgeräumt
            \item \href{http://wiki.piratenpartei.de/Benutzer:LuiseGlobig}{Tätigkeitsbericht im Piratenwiki}
        \end{itemize}
        \item[\getPersonAsWebsiteLink{MiL}:] \     
        \begin{itemize}
            \item Mitgliederverwaltungsportal mit LostInCoding
            \item Trello
        \end{itemize}
        \item[\getPersonAsWebsiteLink{ElH}:] \emph{abwesend}
        \item[\getPersonAsWebsiteLink{FeW}:] \     
        \begin{itemize}
            \item Zugangsdaten Elina und Luise
            \item Mitgliederverwaltungsportal mit Mirco
            \item Trello und Mails
            \item Mobilitätsblogbeitrag
            \item Ankündigung VoSi
        \end{itemize}
    \end{description}
    
    \TOP{Berichte der Beauftragten}
    \emph{-- Noch keine öffentlichen Beauftragungen ausgesprochen. --}
    
    \TOP{Kurzbericht Schatzmeister}
    
    \textbf{Kassenstand}: unverändert:402,00\,\euro
    
    \begin{center}
        \begin{tabular}{|l||r|}
            \hline
            \textbf{Mitglieder gesamt:}             & 33  \\
            \hline
            \textbf{Fördermitglieder:}                & 45  \\                
            \hline
            \hline
        \end{tabular}
    \end{center}
    
    \TOP{Fragen an den Vorstand}
      \emph{keine Fragen.}
      
    \TOP{Termine}
    \begin{description}
        \item[26.--28.01.2018:] Marina Kassel
        \item[14.--15.07.2018:] BAG  Kinderinteressen e.\,V. (Halle/S.)
    \end{description}
    
    \TOP{Anträge an den BuVo}
    \emph{-- Es sind keine Anträge eingegangen. --}
    
    \TOP{AG Programm}
    \begin{itemize}
        \item Terminsuche: Beschließen die Telnehmer individuell
    \end{itemize}
    
    \TOP{Arbeitsweise im Vorstand}
    \begin{itemize}
        \item Verspätete Antwort auf Anfragen
        \item Frage/Diskussion: Wer beantwortet welche Trello-Anfragen? 
        \begin{itemize}
            \item Auftrennung politisch/die Verwaltung betreffend.
            \item Wie werden Tickets aus Mails erzeugt? 
            \item Evtl. ist die Aufteilung in "`Vorstand"' und "`NÖ Vorstand"' zu überarbeiten
            \item Es soll zwei Trello-Boards geben: ein Arbeitsboard und ein Posteingangsboard; beide sind nichtöffentlich (zu erstellen durch LostInCoding; Luise übernimmt die Verwaltung der Mails)
        \end{itemize}
    \end{itemize}
        
    \TOP{Mail-Workflow}
    \begin{itemize}
        \item es soll \emph{eine} Person geben, die die Tickets verwaltet (Separates Posteingangs-Pad)
    \end{itemize}
    
    
            
    \TOP{Einladungen zur Vorstandssitzung}
    \begin{itemize}
        \item möchte LostInCoding übernehmen
        \item Einladung über Kalender, Wordpress und Mailingliste
    \end{itemize}
    
            
    \TOP{Sukzessive Zusammenstellung der Tagesordnung in Trello}
    \begin{itemize}
        \item Problem: Tagesordnung wird immer sehr kurzfristig erstellt. 
        \item Mögliche Lösung: TO im Trello oder in einem Pad erstellen; bis zur nächsten Sitzung testen wir die Trello-Variante
    \end{itemize}
    
            
    
     \TOP{GO-Vorschläge}
    \begin{itemize}
        \item finaler Entwurf \href{https://github.com/JungePiratenDeutschland/Geschaeftsordnung-BV/blob/draft-vs2017_12_20/GO.md}{in github}; soll in der nächsten Sitzung abgestimmt werden.
    \end{itemize}
    
       \TOP{BMV 2018.1?}
    \begin{itemize}
        \item Möglicher Ort: P9 in Berlin
    \end{itemize}
    
    
    \TOP{Vier-Punkte-Plan}
    Auswertung steht noch aus.
    
    \TOP{Sonstiges}
    ---
    
    \TOP{Diskussionen und Anträge}
 
    \naechsteSitzung{03.01.2018}{20:00}

\end{Protokoll}
