
\begin{Protokoll}{
        Sitzungsleitung                 = {JoH}, 			% Kürzel wie in Hauptdatei definiert
        Protokollfuehrer                = {MiL},  			% Kürzel wie in Hauptdatei definiert
        Sitzungszeit/Datum              = {01.11.2017},   	% \
        Sitzungszeit/BeginnZeit         = {20:20},			%  > Die Verwendung dieser Variablen 
        Sitzungszeit/EndeZeit           = {22:42},			% /   wird im Header konfiguriert!
        Ort                             = {Mumble},			% oder jeder andere Ort
        Aufzeichnung/URL                = {},				% gerne auch leer lassen
        Aufzeichnung/Verantwortlicher   = {},		  	    % Kürzel wie in Hauptdatei definiert
        Status                          = {false},			% genehmigt oder nicht (boolean)
    }
    
    \begin{Anwesenheitsliste}
    	\anwesenheit{JoH}{1}
    	\anwesenheit{LuG}{1}
    	\anwesenheit{FeW}{1}
    	\anwesenheit{ElH}{1}
        \anwesenheit{MiL}{1}
    \end{Anwesenheitsliste}
    
	\textbf{Der Vorstand ist beschlussfähig.}
    
    \genehmigeLetztesProtokoll{false}
    Die Abstimmung über das Protokoll wird verschoben.
    
    
    \TOP{Tätigkeitsberichte der Vorstandsmitglieder}
        \begin{description}
            \item[\getPersonAsWebsiteLink{JoH}:] \     
		    	\begin{itemize}
					\item Teilnahme BMV
					\item BPT der Piraten
					\item Tickets
					\item Social Media
					\item Mitgliederaquise
				\end{itemize}
			\item[\getPersonAsWebsiteLink{LuG}:] \     
				\begin{itemize}
					\item Teilnahme BMV
					\item BPT der Piraten
				\end{itemize}
			\item[\getPersonAsWebsiteLink{MiL}:] \     
				\begin{itemize}
					\item Teilnahme an BMV
					\item Besprechung mit LostInCoding
				\end{itemize}
			\item[\getPersonAsWebsiteLink{ElH}:] \     
				\begin{itemize}
					\item  Mumble Team PolGF
				\end{itemize}
			\item[\getPersonAsWebsiteLink{FeW}:] \     
				\begin{itemize}
					\item Vor- und Nachbereitung der BMV
					\item Verarbeitung von Mitgliedsanträgen
					\item Mails lesen und schreiben
					\item Mit Mirco und Schlumpf über eine Software zur Mitgliederverwaltung beraten
					\item Technisches Zeug
				\end{itemize}
		\end{description}
    
    \TOP{Berichte der Beauftragten}
		\emph{-- Noch keine öffentlichen Beauftragungen ausgesprochen. --}
    \TOP{Kurzbericht Bezirkssekretär}
        \begin{center}
            \begin{tabular}{|l||r|}
                \hline
                \textbf{Mitglieder gesamt:}             &  33 \\
                \hline
                \textbf{Fördermitglieder:}				& 45 \\                
                \hline
                \hline
            \end{tabular}
        \end{center}
    
  	\TOP{Fragen an den Vorstand}
  	
  	\begin{description}
  		\item[Wann werden die Jungen Piraten als e. V. eingetragen?]
  			Verzögerung aufgrund formaler Mängel, Postlaufzeiten und Rechtschreibkorrekturen. Momentan liegt die Satzung beim AG und wartet auf Genehmigung.
  		\item[Existiert ein Konto?]
  			Momentan noch nicht.\\ \emph{Nach Anfrage eines Gastes:} Momentan wird in GnuCash gebucht
  		\item[Ist die ladungsfähige Anschrift auf der Website korrekt?]
  			Momentan ist angedacht, dass ein Postfach in der Geschäftsstelle der Piratenpartei Deutschland eingerichtet wird; das Thema ist folglich noch in der Schwebe.
  			Die Sulzbecker Straße kann momentan verwendet werden. Postfach ist keine ladungsfähige Anschrift.
  		\item[Anfrage vom Staatsarchiv Bremen]
  			Das Staatsarchiv Bremen hat sich auf politische Parteien spezialisiert und möchte daher die Publikationen der JuPis bekommen.
  	\end{description}
    
    \TOP{Termine}
	    \begin{itemize}
	    	\item 15.10.2017 LTW Niedersachsen
	    	\item 21./22.10.2017 BPT Regensburg
	    	\item 22.10.2017 BMV Regensburg
	    	\item 11./12.11.2017 LPT Bayern Lauf an Pegnitz
	    \end{itemize}
    
    \TOP{Anträge an den BuVo}
    	\emph{-- Es sind keine Anträge eingegangen. --}
    
    \TOP{AG Programm}
    	\begin{itemize}
    		\item wenig besucht
    		\item mehr Motivation
    		\item Im Team PolGf ankündigen für mehr Interessenten (auch Piraten und Nicht-JuPis)
    	\end{itemize}
    
    \TOP{BMV \&\ BPT}
    	\begin{description}
    		\item[Terminfindungsverfahren-Diskussion:]
    			Am BPT oder abseits (eventuell dennoch zu Parteiveranstaltung?) $\rightarrow$ Eventuell im Wechsel
    		\item[Anträge zum BPT:]
    		SÄA\,003, bisher keine Einigkeit über JuPi-Fraktion im potentiellen kleinen BPT (bei Annahme auf jeden Fall nochmals darüber reden)
    	\end{description}
    
    \TOP{Vier-Punkte-Plan}
    	\begin{itemize}
    		\item Lime Survey 
    		\item Umfrage $\rightarrow$ Mumble $\rightarrow$ Vorstandklausur $\rightarrow$ Back to Basis
    		\item Nutzung von Lime Survey: Umfrageportal mit individueller Einladungsmail (keine weitere, seperate Registrierung erforderlich)
    	\end{itemize}
    
    \TOP{Sonstiges}
    	---
    
    \TOP{Diskussionen und Anträge}
       \Beschluss{
            Umlaufbeschluss     = {false},
            Antragsteller       = {MiL},			% Kürzel wie in Hauptdatei definiert
            Thema               = {Änderung der Geschäftsordnung},
            Gesamtwert          = {1.500},			% €-Zeichen wird automatisch ergänzt
            Antragstext         = {Der Vorstand möge beschließen, § 1 seiner Geschäftsordnung durch folgende Fassung zu ersetzen: \\
            	"`Social Media: Jonathan-Benedict Hütter; LuiseGlobig (stellvertretend) \\
            	Presse \&\ Webseitencontent: Jonathan-Benedict Hütter; LuiseGlobig (stellvertretend) \\
            	Externe Kommunikation (andere Vereine und Jugendorganisationen): Jonathan-Benedict Hütter; LuiseGlobig (stellvertretend) \\
            	International cooperations/Internationale Beziehungen: LuiseGlobig; Jonathan-Benedict Hütter (stellvertretend) \\
            	Kontakt zur Piratenpartei: Elina Hattendorf; LuiseGlobig und Jonathan-Benedict Hütter (stellvertretend) \\
            	Mitgliederbetreuung: LuiseGlobig; Elina (stellvertretend) \\
            	Finanzen, Verwaltung und IT: Felix Wöstmann; Mirco Lukas (stellvertretend)\\
            	Organisation und Pflege des Ticketsystems: Mirco Lukas; Felix Wöstmann (stellvertretend)"'},
            Begruendung         = {},
            Verantwortlicher    = {FeW},			% Kürzel wie in Hauptdatei definiert
            Abstimmung          = {JoH/1,LuG/1,FeW/1,ElH/1,MiL/1},
            										% Kürzel wie in Hauptdatei definiert;
            										% Werte: 1, 0, e(nthaltung), a(bwesend), 
            										%        x (noch nicht abgestimmt)
            Endergebnis         = {1}, 				% Werte: 1, 0, z(urückgezogen), 
            										%        x (in Abstimmung)
            Bemerkungen         = {}
        }
    
    \Beschluss{
    	Umlaufbeschluss     = {false},
    	Antragsteller       = {JoH},			% Kürzel wie in Hauptdatei definiert
    	Thema               = {Änderung der Geschäftsordnung},
    	Gesamtwert          = {1.500},			% €-Zeichen wird automatisch ergänzt
    	Antragstext         = {Der Vorstand möge beschließen, folgenden Paragraphen als §\,6 in die Geschäftsordnung einzufügen:
    		"`Redaktionelle Änderungen dieser Geschäftsordnung bedürfen keines Vorstandsbeschlusses."'},
    	Begruendung         = {},
    	Verantwortlicher    = {FeW},			% Kürzel wie in Hauptdatei definiert
    	Abstimmung          = {JoH/1,LuG/1,FeW/1,ElH/1,MiL/0},
    	% Kürzel wie in Hauptdatei definiert;
    	% Werte: 1, 0, e(nthaltung), a(bwesend), 
    	%        x (noch nicht abgestimmt)
    	Endergebnis         = {z}, 				% Werte: 1, 0, z(urückgezogen), 
    	%        x (in Abstimmung)
    	Bemerkungen         = {}
    }

	\TOP{Nächster Termin}
		22.11.2017, 20:00\,Uhr.
\end{Protokoll}

