
\begin{Protokoll}{
        Sitzungsleitung                 = {LuG},             % Kürzel wie in Hauptdatei definiert
        Protokollfuehrer                = {MiL},              % Kürzel wie in Hauptdatei definiert
        Sitzungszeit/Datum              = {03.01.2018},       % \
        Sitzungszeit/BeginnZeit         = {20:00},            %  > Die Verwendung dieser Variablen 
        Sitzungszeit/EndeZeit           = {21:20},            % /   wird im Header konfiguriert!
        Ort                             = {Mumble},            % oder jeder andere Ort
        Aufzeichnung/URL                = {},                % gerne auch leer lassen
        Aufzeichnung/Verantwortlicher   = {MiL},                  % Kürzel wie in Hauptdatei definiert
        Status                          = {true},            % genehmigt oder nicht (boolean)
    }
    
    \begin{Anwesenheitsliste}
        \anwesenheit{JoH}{1}
        \anwesenheit{LuG}{1}
        \anwesenheit{FeW}{1}
        \anwesenheit{ElH}{1}
        \anwesenheit{MiL}{1}
    \end{Anwesenheitsliste}
    
    \textbf{Der Vorstand ist beschlussfähig.}
    
    \genehmigeLetztesProtokoll{true} \\
    Das {Protokolls der ersten Sitzung} wurde genehmigt.
    
    % https://github.com/JungePiratenDeutschland/Protokoll/blob/master/2017_Vorstand/Protokoll.pdf
    
    \TOP{Tätigkeitsberichte der Vorstandsmitglieder}
    \begin{description}
        \item[\getPersonAsWebsiteLink{JoH}:] \     
        \begin{itemize}
            \item Arbeit mit Trello
        \end{itemize}
        \item[\getPersonAsWebsiteLink{LuG}:] \     
        \begin{itemize}
            \item Mumble
            \item BuVoSitzung JuPis
             \item AG Programm
             \item Thementreffen Newsletter/Online Stammtisch/Mails mit Felix
             \item Trello
            \item Mails und Telefonate
            \item Einladung AG Programm
        \end{itemize}
        \item[\getPersonAsWebsiteLink{MiL}:] \     
        \begin{itemize}
            \item Organisatorisches
            \item Kontakt mit GeldPirat für Finzanzverwaltungssoftware aufgenommen
        \end{itemize}
        \item[\getPersonAsWebsiteLink{ElH}:] \     
        \begin{itemize}
            \item Versmmlung der Kreispiraten in Wolfenbüttel
            \item Teilnahme am 34C3
            \item Teilnahme an PolGF-Sitzungen
        \end{itemize}
        \item[\getPersonAsWebsiteLink{FeW}:] \     
        \begin{itemize}
            \item Mumble mit Luise zu verschiedenen Themen
            \item Umstellen des Trellos inklusive Mails
            \item Trello Administration
            \item Zugänge zurücksetzen, Änderungen vornehmen
        \end{itemize}
    \end{description}
    
    \TOP{Berichte der Beauftragten}
        \emph{-- Noch keine öffentlichen Beauftragungen ausgesprochen. --}
    
    \TOP{Kurzbericht Schatzmeister}
    
    \textbf{Kassenstand}: 402,00\,\euro
    
    \begin{center}
        \begin{tabular}{|l||r|}
            \hline
            \textbf{Mitglieder gesamt:}             & 33  \\
            \hline
            \textbf{Fördermitglieder:}                & 45 \\                
            \hline
            \hline
        \end{tabular}
    \end{center}
     
    \TOP{Fragen an den Vorstand}
    
    \begin{description}
        \item
    \end{description}
    
    \TOP{Termine}
    \begin{itemize}
        \item  Es existiert bereits ein Pad für die Terminsammlung.
    \end{itemize}
    
    \TOP{Anträge an den BuVo}
    \emph{-- Es sind keine Anträge eingegangen. --}
    
    \TOP{AG Programm}
    \begin{itemize}
        \item 
    \end{itemize}
        
    \TOP{Vier-Punkte-Plan}
    \begin{itemize}
        \item Keine Neuigkeiten.
        \item Daten zur Auswertung liegen bei Jojo; Auswertung soll innerhalb der nächsten zwei Wochen geschehen.
    \end{itemize}

    \TOP{Jupi-Camp}
    \begin{itemize}
        \item wurde wegen mangelndem Interesse nicht weiter verfolgt.
        \item Diskussion: sollte ein weiteres Camp stattfinden?
        \item Vorschlag: OpenMind (Vorträge von engagierten Jupis)
    \end{itemize}
    
    \TOP{Newsletter}
    \begin{itemize}
        \item Luise und Felix: Newsletter zum 01.02.: Auf Webseite, per E-Mail und per Abo an  Mitglieder und Interessierte.
        \item Inhalt: auch Termine; daher sollten diese im Pad8 \url{https://vorstandsteam-junge-piraten.piratenpad.de/Termine} veröffentlicht werden.
    \end{itemize}

    \TOP{Online-Stammtisch}
    \begin{itemize}
        \item soll jeweils eine Woche nach der Bundes-Vorstandssitzung stattfinden (erstmalig am 24.01.2017)
        \item Name: "Auf ein Wort - Der digitale Stammtisch der Jungen Piraten" 
        \item Einladung in PolGF-Sitzung 
    \end{itemize}

    \TOP{Trello und Mails}
    \begin{itemize}
        \item Vorschlag: Zusammenführung oder Umbenennung der Trello-Pads "NÖ Vorstand" und "Vorstand".
        \item Neu: Neue E-Mails gehen an Trello-Board "Posteingang".
    \end{itemize}
   
    \TOP{Sonstiges}
    ---
    
    \TOP{Diskussionen und Anträge}
    
       \Beschluss{
        Umlaufbeschluss     = {false},
        Antragsteller       = {JoH},            % Kürzel wie in Hauptdatei definiert
        Thema               = {Ohne GO arbeiten},
        Gesamtwert          = {0},            % €-Zeichen wird automatisch ergänzt
        Antragstext         = {Der Vorstand möge beschließen, bis auf Weiteres ohne Geschäftsordnung weiterzuarbeiten.},
        Begruendung         = {Wir haben jetzt fast 3 Monate ohne GO gebraucht und es wird voraussichtlich zur nächsten BMV im Frühjahr erneute Personalwechsel geben.},
        Verantwortlicher    = {FeW},          
        Abstimmung          = {JoH/e,LuG/0,FeW/0,ElH/0,MiL/0},
        Endergebnis         = {0}, 
        Bemerkungen         = {}
    }
    
    \Beschluss{
        Umlaufbeschluss     = {false},
        Antragsteller       = {LuG},            % Kürzel wie in Hauptdatei definiert
        Thema               = {Änderung der GO},
        Gesamtwert          = {0},            % €-Zeichen wird automatisch ergänzt
        Antragstext         = {Der Vorstand möge beschließen, die Fassung der GO \url{https://raw.githubusercontent.com/JungePiratenDeutschland/Geschaeftsordnung-BV/draft-vs2017_12_20/GO.md} anzunehmen (Commit f6b4795).},
        Begruendung         = {},
        Verantwortlicher    = {MiL},            % Kürzel wie in Hauptdatei definiert
        Abstimmung          = {JoH/e,LuG/1,FeW/1,ElH/1,MiL/1},
        Endergebnis         = {1},
        Bemerkungen         = {}
    }
    
  
     \naechsteSitzung{17.01.2018}{20:00}
    
\end{Protokoll}


