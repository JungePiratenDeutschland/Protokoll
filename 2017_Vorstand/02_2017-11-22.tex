\begin{Protokoll}{
        Sitzungsleitung                 = {MiL},             % Kürzel wie in Hauptdatei definiert
        Protokollfuehrer                = {MiL},              % Kürzel wie in Hauptdatei definiert
        Sitzungszeit/Datum              = {22.11.2017},       % \
        Sitzungszeit/BeginnZeit         = {20:00},            %  > Die Verwendung dieser Variablen 
        Sitzungszeit/EndeZeit           = {20:55},            % /   wird im Header konfiguriert!
        Ort                             = {Mumble},            % oder jeder andere Ort
        Aufzeichnung/URL                = {},                % gerne auch leer lassen
        Aufzeichnung/Verantwortlicher   = {},                  % Kürzel wie in Hauptdatei definiert
        Status                          = {false},            % genehmigt oder nicht (boolean)
    }
    
\newPerson{Mel}{
    Name = {Melano},
   	Amt = {\emph{Gast}}
}

    \begin{Anwesenheitsliste}
        \anwesenheit{JoH}{1}
        \anwesenheit{LuG}{1}
        \anwesenheit{FeW}{1}
        \anwesenheit{ElH}{1}
        \anwesenheit{MiL}{1}
        \anwesenheit{Mel}{1}
    \end{Anwesenheitsliste}
    
    \textbf{Der Vorstand ist beschlussfähig.}
    
    \genehmigeLetztesProtokoll{false}
    
    
    \TOP{Tätigkeitsberichte der Vorstandsmitglieder}
    \begin{description}
        \item[\getPersonAsWebsiteLink{JoH}:] \     
        \begin{itemize}
            \item Umfrage erstellt und versendet
            \item Teilnahmen an PolGF-Sitzung
        \end{itemize}
        \item[\getPersonAsWebsiteLink{LuG}:] \     
        \begin{itemize}
            \item Teilnahme an PolGF-Sitzung der Piratenpartei
            \item Planung für zukünftige Arbeit
        \end{itemize}
        \item[\getPersonAsWebsiteLink{MiL}:] \     
        \begin{itemize}
           \item Neuer Entwurf der GO:
                 \begin{description}
                           \item[Nur Tippfehlerkorrekturen:]  \url{https://github.com/JungePiratenDeutschland/Geschaeftsordnung-BV/blob/master/GO.md}
                           \item[Überarbeitete Fassung:] \url{https://github.com/JungePiratenDeutschland/Geschaeftsordnung-BV/blob/draft/GO.md}
                    \end{description}
           \item Neuer Entwurf des Protokolls: \url{https://github.com/JungePiratenDeutschland/Protokoll/raw/master/2017_Vorstand/Protokoll.pdf}
           \item Aufräumen der Satzung (nur Markdown) und der Homepage
            \item Mit LostInCoding: 
                  \begin{itemize}
                           \item Einführung  in JuPi-Verwaltung
                           \item Planungen für die zukünftige Mitgliederverwaltung
                    \end{itemize}
        \end{itemize}
        \item[\getPersonAsWebsiteLink{ElH}:] \     
        \begin{itemize}
            \item Teilnahme an PolGF-Sitzungen
            \item \emph{Zwischenfrage:} Wie sieht es mit Neuwahlen bei den Piraten aus?
        \end{itemize}
        \item[\getPersonAsWebsiteLink{FeW}:] \     
        \begin{itemize}
            \item Mumble-Besprechung mit Mirco: Technik, Zugänge, NextCloud, GO-Entwurf, Mitgliederverwaltung
            \item Daten in zentrale Verwaltungsdatei verschoben
            \item Satzungschronologie im Wiki angepasst
            \item Wordpress dokumentiert
            \item GitHub dokumentiert
            \item NextCloud dokumentiert
        \end{itemize}
    \end{description}
    
    \TOP{Berichte der Beauftragten}
        \emph{-- Noch keine öffentlichen Beauftragungen ausgesprochen. --}
    
    \newpage
    
    \TOP{Kurzbericht Schatzmeister}
    
    \textbf{Kassenstand:} 402,00\,\euro \\
    
    \noindent\textbf{Mitgliederentwicklung:}
    \begin{center}
        \begin{tabular}{|l||r|}
            \hline
            \textbf{Mitglieder gesamt:}             & 33  \\
            \hline
            \textbf{Fördermitglieder:}               & 45 \\                
            \hline
            \hline
        \end{tabular}
    \end{center}
    
    \emph{Hinweis:} Es haben weniger Mitglieder eine Mail zur Umfrage erhalten, da nicht alle ihre E-Mail-Adresse angegeben haben.
    
    \TOP{Fragen an den Vorstand}
      \begin{itemize}
        \item Generelle Frage: Sollen Termine der Piraten SH, die Jugendarbeit betreffen, an uns kommuniziert werden? \\
        \emph{-- Ja, bitte ein Ticket an hallo@junge-piraten.de}
    \end{itemize}
    
    \TOP{Termine}

    \begin{itemize}
        \item \emph{25.11.2017}: Ständemeile der Münchner Schüler
    \end{itemize}
    
    \TOP{Anträge an den BuVo}
    \emph{-- Es sind keine Anträge eingegangen. --}
    
    \TOP{AG Programm}
    \begin{itemize}
        \item[Luise und Luna:] gemeinsame Arbeit besprochen;
          \begin{description}
        \item[Modusplanung:] regelmäßige Termine, Einladung in PolGF-Sitzungen
        \item[Termin:] jeweisl eine Woche nach den Vorstandssitzungen; gleicher Tag, gleiche Zeit
        \end{description}
    \end{itemize}
    
    \TOP{Vier-Punkte-Plan}
    \begin{itemize}
        \item Umfrage \emph{(erledigt)}
        \item Vorstandsklausur \emph{(ausstehend; vstl. Q1/2018)}
        \item BMV \emph{(ausstehend)}
    \end{itemize}
    
    \TOP{Sonstiges}
       \begin{itemize}
        \item Anregung: Arbeitsseite, in der man z.\,B. fehlende Zugänge monieren kann \\
            \emph{-- existiert bereits im Wiki, Link folgt}
        \item Mehr Beteiligung für Social Media, insbesondere bei Telegram-Anfragen
     \end{itemize}
     
     
    \TOP{Diskussionen und Anträge}
  		\emph{-- Es liegen keine Anträge vor. --}
    
    
    \TOP{Nächster Termin}
		Der nächste Termin wird per Umlauf (Doodle) beschlossen.
\end{Protokoll}


